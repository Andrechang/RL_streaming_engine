NDC architecture contains a Coarse Grain Reconfigurable Array (CGRA) Streaming Engine (SE). 
It is a custom hardware architecture that provides programming flexibility that enables high-performance and power savings. 
A program is represented as a computation graph and its operations need to be properly mapped into the SE device to ensure correct execution and dataflow. 
This creates an optimization problem with a vast and sparse search space. Creating mappings manually or using brute force algorithms mapping takes time and lots of effort. 
It also adds assumptions that reduce search-space trading off optimization possibilities. Programing SE manually requires expertise on how all the SE operates, which adds entry barrier to use SE device. 
In this work we propose a learning method to explore and optimize in an unsupervised manner using reinforcement learning (RL) framework. 
This provides an automated method that can produce programs quickly and searches for optimal mappings without programmers’ interference. 
This tool also improves the usability of the SE device by encapsulating device configuration details. 
Proximal Policy Optimization (PPO) training a model to place operations into the SE tiles based on a reward function that models the SE device and its constraints. 
Graph neural networks are added to create embeddings to represent the computation graph.
 A transformer block is used to model sequential operation placement mode. 
 The trained model was able to create valid mappings for SE. 
 The implemented method is compared against evolutionary search and baseline. 